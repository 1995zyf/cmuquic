\documentclass{article}
\usepackage{amsmath}
\usepackage{comment}
\usepackage{graphicx}
\usepackage{hyperref}
\author{Spencer Baugh (sbaugh), Xingda Zhai (xingdaz)}
\title{Project Proposal: Fast, Production-Ready Implementation of QUIC}
\date{\today}
\begin{document}
\maketitle

\section{Problem Description}

QUIC is a new transport protocol for the web, developed by Google.
It is implemented on top of UDP, and it replaces TCP and TLS with a single protocol.
It shares several features with SPDY,
a protocol developed by Google for the purpose of speeding up HTTP connections, which was layered on top of TCP and TLS.
However, unlike SPDY, QUIC replaces TCP and TLS, reimplementing the necessary features.
QUIC is an ``application-level protocol'', and is not intended to be deployed in kernel space;
this allows bypassing the slow pace of kernel development.

Key advantages of QUIC over TCP+TLS+SPDY include:
\begin{itemize}
\item Reduced connection establishment latency
\item Improved congestion control
\item Multiplexing without head-of-line blocking
\item Forward error correction
\item Connection migration between links
\end{itemize}

Google has been running its QUIC implementation for traffic between the Chrome browser and its servers
and has reported drastic performance improvements for web and especially Youtube videos:
for page loading, 3% faster and 30% fewer video rebuffers.

While the client-side implementation of QUIC in Chrome is open source and production quality,
there is only a toy server example out in the wild.
Without a high quality and fast open source implementation,
QUIC adoption is stalled.

Therefore, we will implement a high performance QUIC server
using the Data Plane Development Toolkit (DPDK),
a set of libraries and drivers useful for implementing fast packet processing on Intel CPUs.
With the speed of DPDK and the architectural improvements of QUIC,
we hope our implementation will compare favorably to the traditional TCP+TLS+HTTP stack,
and even compete with TCP+TLS+SPDY or TCP+TLS+HTTP2.

\section{Milestones}

\begin{enumerate}
\item Research the QUIC protocol, assembling the documents we will use as a reference while implementing it.
  \begin{itemize}
  \item 
    \href{https://tools.ietf.org/html/draft-tsvwg-quic-protocol-02}{Draft RFC for the QUIC protocol}
  \item 
    \href{https://tools.ietf.org/html/draft-quic-loss-recovery}{Draft RFC for the QUIC protocol's loss recovery and congestion control techniques}
  \item 
    \href{https://docs.google.com/document/d/1g5nIXAIkN_Y-7XJW5K45IblHd_L2f5LTaDUDwvZ5L6g/}{QUIC protocol documentation}
  \end{itemize}
\item
  Examine the implementation and design decisions in the 
  \href{https://chromium.googlesource.com/chromium/src/+/master/net/tools/quic/}{toy server and client} 
  provided by Google.
\item 
  Familiarize ourselves with \href{http://dpdk.org/doc}{DPDK}.
  As neither of us have any pre-existing experience with DPDK,
  at this or some later stage we may, with the benefit of more information, decide to use some other high-performance networking toolkit.
\item 
  Reimplement QUIC, without the QUIC crypto protocol, with DPDK.
  This is interesting in its own right,
  since the QUIC crypto protocol, in the words of the protocol authors, ``is destined to die'', i.e. it will be replaced by TLS 1.3.
\item 
  Modify the Google-provided toy client to test this version of QUIC.
\item 
  Add the QUIC crypto protocol to our implementation.
\item 
  Test our new implementation with the official Google version of the toy client, as well as Chromium.
\item 
  Benchmark our new implementation, and compare it to TCP+TLS+HTTP, TCP+TLS+SPDY, and TCP+TLS+HTTP2, as well as Google's toy server.
  These comparison will be carried out across multiple networks, including Ethernet, WiFi, and cellular,
  under simulated congestion or latency,
  and under conditions of mobility between networks.
\end{enumerate}

\section{Deliverables}

We will endeavor to produce a fully functional, production grade QUIC server implemented with DPDK or an equivalent high-performance networking toolkit.
We also plan to compare the performance of our server to the available alternative web stacks, as described above.

If we have additional time,
further work might include:
\begin{itemize}
\item The integration of our implementation into nginx.
\item An investigation of the behavior of Google's production servers, and a comparison of our server thereto.
\end{itemize}
\end{document}
